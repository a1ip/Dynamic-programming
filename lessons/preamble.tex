\usepackage[
  a4paper, mag=1000,
  left=2cm, right=1cm, top=2cm, bottom=2cm, headsep=0.7cm, footskip=1.27cm
]{geometry}

\usepackage[T2A]{fontenc}
\usepackage[utf8]{inputenc}
\usepackage{ucs}
\usepackage[english,russian]{babel}
\usepackage{cmap}
\usepackage{pscyr} % Нормальные шрифты
\usepackage{amsmath}
\usepackage{tabularx}
\usepackage{array}
\usepackage{multirow}
\usepackage{amssymb}

% цвет в документе
\usepackage[usenames]{color}
\usepackage{colortbl}

% собственные разделы
\usepackage{titlesec}
\usepackage{titletoc}

% Форматирование чисел
\usepackage{siunitx}
\sisetup{range-phrase = \ldots}

% оглавление + ссылки
\usepackage[pdftex]{hyperref}

\hypersetup{%
    pdfborder = {0 0 0},
    colorlinks,
    citecolor=red,
    filecolor=Darkgreen,
    linkcolor=blue,
    urlcolor=blue
}

\usepackage{hypcap}
\usepackage[numbered]{bookmark} % для цифр в закладках в pdf viewer'ов

\usepackage{tocloft}

\renewcommand{\cftsecaftersnum}{.}
\renewcommand{\cftsubsecaftersnum}{.}

\setcounter{tocdepth}{4}
\setcounter{secnumdepth}{4}

\renewcommand\cftsecleader{\cftdotfill{\cftdotsep}}
\renewcommand{\cfttoctitlefont}{\hfill\Large\bfseries}
\renewcommand{\cftaftertoctitle}{\hfill}

% листинги
\usepackage{listings}
\usepackage{color}
\definecolor{commentGreen}{rgb}{0,0.6,0}
\usepackage{caption} % для подписей под листингами и таблиц
%\usepackage[scaled]{beramono}

% графика
\usepackage[pdftex]{graphicx}
\graphicspath{{./fig/}} % папка с изображениями
\usepackage[tikz]{bclogo}
\usepackage{wrapfig}

%\usepackage{setspace} % для полуторного интервала
%\onehalfspacing % сам полуторный интервал

\usepackage{indentfirst} % отступ в первом абзаце

% точки в секция и подсекциях
\usepackage{secdot}
\sectiondot{subsection}

% Подавление висячих строк
\clubpenalty=10000
\widowpenalty=10000

% колонтитулы
%\usepackage{fancybox,fancyhdr}
%\pagestyle{fancy}
%\fancyhf{}
%\fancyhead[C]{\small{C++ and Qt}} % шапка верхнего колонтитула!!!
%\fancyfoot[C]{\small{\thepage}} % шапка нижнего колонтитула!!!

% стили листингов кодов

\lstdefinestyle{CPlusPlus}{
  language=C++,
  basicstyle=\small\ttfamily,
  breakatwhitespace=true,
  breaklines=true,
  showstringspaces=false,
  keywordstyle=\color{blue}\ttfamily,
  stringstyle=\color{red}\ttfamily,
  commentstyle=\color{commentGreen}\ttfamily,
  morecomment=[l][\color{magenta}]{\#},
  numbers=left,
  xleftmargin=1cm
}
\lstset{style=CPlusPlus}
\lstset{inputencoding=utf8, extendedchars=\true}
\DeclareUnicodeCharacter{00A0}{~}

%% новый раздел "занятие"
%\titleclass{\lesson}{straight}[\part]
%\newcounter{lesson}
%\renewcommand{\thelesson}{\arabic{lesson}}
%
%\titleformat{\lesson}[display]
%  {\normalfont\Huge\bfseries}{\huge Занятие \thelesson}{20pt}{}
%\titlespacing*{\lesson}{0pt}{50pt}{40pt}
%
%\titlecontents{lesson}
%    [1.5em] % ie, 1.5em (chapter) + 2.3em
%    {}
%    {\bfseries\contentslabel{1.5em}}
%    {}
%    {\bfseries\titlerule*[1pc]{}\contentspage}

% Задачи в тексте
%\newenvironment{task}
%    { \begin{tabular}{|p{18cm}|} \hline }
%    { \\ \hline \end{tabular} }
\newenvironment{task}
 { \vspace{2ex}\hrule\vspace{2ex}$\bigstar$ }
 { \vspace{2ex}\hrule\vspace{2ex} }

\newcommand{\stirling}[2]{\left\{{#1}\atop{#2}\right\}}
\newcommand{\euler}[2]{\left<{#1}\atop{#2}\right>}
