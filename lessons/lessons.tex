\documentclass[14pt]{book}

\usepackage[
  a4paper, mag=1000,
  left=2cm, right=1cm, top=2cm, bottom=2cm, headsep=0.7cm, footskip=1.27cm
]{geometry}

\usepackage[T2A]{fontenc}
\usepackage[utf8]{inputenc}
\usepackage[english,russian]{babel}
\usepackage{cmap}
\usepackage{pscyr} % Нормальные шрифты
\usepackage{amsmath}
\usepackage{tabularx}
\usepackage{array}
\usepackage{multirow}
\usepackage{amssymb}

% цвет в документе
\usepackage[usenames]{color}
\usepackage{colortbl}


% Форматирование чисел
\usepackage{siunitx}
\sisetup{range-phrase = \ldots}

% оглавление + ссылки
\usepackage[pdftex]{hyperref}

\hypersetup{%
    pdfborder = {0 0 0},
    colorlinks,
    citecolor=red,
    filecolor=Darkgreen,
    linkcolor=blue,
    urlcolor=blue
}

\usepackage{hypcap}
\usepackage[numbered]{bookmark} % для цифр в закладках в pdf viewer'ов

\usepackage{tocloft}

\renewcommand{\cftsecaftersnum}{.}
\renewcommand{\cftsubsecaftersnum}{.}

\setcounter{tocdepth}{4}
\setcounter{secnumdepth}{4}

\renewcommand\cftsecleader{\cftdotfill{\cftdotsep}}
\renewcommand{\cfttoctitlefont}{\hfill\Large\bfseries}
\renewcommand{\cftaftertoctitle}{\hfill}


% листинги
\usepackage{listings}
\usepackage{color}
\definecolor{commentGreen}{rgb}{0,0.6,0}
\usepackage{caption} % для подписей под листингами и таблиц
%\usepackage[scaled]{beramono}

% графика
\usepackage[pdftex]{graphicx}
\graphicspath{{./fig/}} % папка с изображениями
\usepackage[tikz]{bclogo}
\usepackage{wrapfig}

%\usepackage{setspace} % для полуторного интервала
%\onehalfspacing % сам полуторный интервал

\usepackage{indentfirst} % отступ в первом абзаце

% точки в секция и подсекциях
\usepackage{secdot}
\sectiondot{subsection}

% Подавление висячих строк
\clubpenalty=10000
\widowpenalty=10000

% колонтитулы
%\usepackage{fancybox,fancyhdr}
%\pagestyle{fancy}
%\fancyhf{}
%\fancyhead[C]{\small{C++ and Qt}} % шапка верхнего колонтитула!!!
%\fancyfoot[C]{\small{\thepage}} % шапка нижнего колонтитула!!!

% стили листингов кодов

\lstdefinestyle{CPlusPlus}{
  language=C++,
  basicstyle=\small\ttfamily,
  breakatwhitespace=true,
  breaklines=true,
  showstringspaces=false,
  keywordstyle=\color{blue}\ttfamily,
  stringstyle=\color{red}\ttfamily,
  commentstyle=\color{commentGreen}\ttfamily,
  morecomment=[l][\color{magenta}]{\#},
  numbers=left,
  xleftmargin=1cm
}

% Задачи в тексте
%\newenvironment{task}
%    { \begin{tabular}{|p{18cm}|} \hline }
%    { \\ \hline \end{tabular} }
\newenvironment{task}
 { \vspace{2ex}\hrule\vspace{2ex}$\bigstar$ }
 { \vspace{2ex}\hrule\vspace{2ex} }

\newcommand{\stirling}[2]{\left\{{#1}\atop{#2}\right\}}
\newcommand{\euler}[2]{\left<{#1}\atop{#2}\right>}

\begin{document}

\begingroup
\hypersetup{linkcolor=black}
\tableofcontents
\endgroup

\clearpage

\chapter{Рекуррентные соотношения}


\section{Ханойская башня}

Ханойская башня --- несколько дисков, нанизанных в порядке уменьшения размеров на один из
трех колышков. Задача состоит в том, чтобы переместить всю башню на один из других колышков,
перенося каждый раз только один диск и не помещая больший диск на меньший.
Какое количество перемещений дисков необходимо и достаточно для выполнения задачи?

\textsc{Обозначай и властвуй}: путь $T_n$ --- минимальное число перекладываний, необходимых
для перемещения башни размером $n$ с одного колышка на другой. Очевидно, что $T_1=1$, а $T_2=3$.
Специальный крайний случай --- $T_0=0$, ведь для перемещения башни без дисков вообще ничего делать
не нужно.

Как переместить всю бащню? Эксперементы с тремя дисками показывают, что для такой башни нужно 
сначала переместить два верхних диска на один колышек, потом нижний диск на другой. И уже 
после этого два верхних диска снова перемещаются на самый большой. В общем случае мы сначала
перемещаем $n-1$ меньших дисков на любой из колышков (что требует $T_{n-1}$ перекладываний),
затем перемещаем самый большой диск (одно перекладывание) и, наконец, помещаем $n-1$
меньших дисков обратно на самый большой диск (ещё $T_{n-1}$ перекладываний). 

Таким образом, $n$ дисков (при $n>0$) можно переместить самое большее за $2T_{n-1}+1$
перекладываний. Меньше перекладываний сделать не получится, потому что когда мы хотим
переместить самый большой диск, $n-1$ меньших дисков должны находиться на одном колышке,
а для этого потребуется не меньше $T_{n-1}$ перекладываний.
Несколько раз перекладывать большой диск смысла нет. После его перемещения меньшие диски нужно
переместить на него обратно за $T_{n-1}$ перекладываний. Следовательно,

\begin{equation}
\label{formula-hanoi-rec}
\begin{array}{ll}
T_0 = 0, \\
T_n = 2T_{n-1} + 1 & \text{при } n > 0 \\
\end{array}
\end{equation}

Совокупность равенств типа~\ref{formula-hanoi-rec} называется \emph{рекуррентностью}. Она задается
начальным значением и зависимостью общего члена от предыдущих.

Рекуррентность вида~\ref{formula-hanoi-rec} позволяет вычислять $T_n$ для любого $n$, какое мы пожелаем. Но в действительности
никто не захочет пользоваться для вычисления рекуррентностью, когда $n$ велико --- это займёт
слишком много времени. Рекуррентность даёт только косвенную, локальную информацию.
Решение рекуррентного соотношения --- это запись $T_n$ в простой и компактной
<<замкнутой форме>>. Такая запись позволяет вычислить $T_n$ быстро даже при большом $n$.

Прибавим $1$ к обеим частям уравнения:

\begin{equation}
\begin{array}{ll}
T_0 + 1 = 1, \\
T_n + 1 = 2T_{n-1} + 2 & \text{при } n > 0 \\
\end{array}
\end{equation}

Теперь, если положить $U_n = T_n + 1$, то получим 

\begin{equation}
\begin{array}{ll}
U_0 = 1, \\
U_n = 2U_{n-1} & \text{при } n > 0 \\
\end{array}
\end{equation}

Очевидно, что $U_n = 2^n$. Тогда $T_n = 2^n - 1$.


\subsection{Метод математической индукции}

Значения $T_n$ можно вычислять для разных $n$. Получится $T_0=0, T_1=1, T_2=3, T_3=7, T_4=15$.
Можно предположить, что $T_n=2^n-1$. Проверим это с помощью математической индукции.

База индукции тривиальна --- $T_0 = 0$. Индуктивный переход: если $T_{n-1}=2^{n-1}-1$, то
$T_n = 2T_{n-1}+1 = 2(2^{n-1}-1)+1 = 2^n - 1$. Следовательно, предположение было справедливо.

\section{Задача о разрезании пиццы}

Сколько кусков пиццы можно получить, делая $n$ прямолинейных разрезов ножом?
Или же, каково максимальное число $L_n$ областей, на которые плоскость делится
$n$ прямыми?

Рассмотрим крайние случаи. Плоскость без прямых --- это одна область, 
с одной прямой --- две области, а с двумя прямыми --- четыре области.
Может показаться, что $L_n = 2^n$, но в общем случае это не так.
При добавлении третьей прямой мы можем пересечь самое большее три области,
поэтому $L_3 = 7$.

Новая $n$-я прямая увеличивает число областей на $k$, когда рассекает $k$ старых
областей. А рассекает она $k$ старых областей, когда пересекает прежние прямые
в $k-1$ различных местах. Две прямые могут пересекаться не более чем в одной точке.
Поэтому новая прямая может пересекать $n-1$ старых прямых не более чем в $n-1$
различных точках. Если проводить прямую так, чтобы она не была параллельна ни одной
из уже имеющихся, она обязательно пересечет их все. При этом, надо выбирать прямую так,
чтобы она не проходила через уже имеющиеся пересечения. Поэтому искомое рекуррентное
соотношение имеет вид:

\begin{equation}
\label{formula-pizza-rec}
\begin{array}{ll}
L_0 = 1, \\
L_n = L_{n-1} + n & \text{при } n > 0 \\
\end{array}
\end{equation}

Теперь можно вычислить значения этого ряда: $1, 2, 4, 7, 11, 16, \ldots$. 
Но теперь довольно сложно угадать что за замкнутая форма стоит за ними.

\subsection{Разворачивание рекуррентности}

Развернем формулу~\ref{formula-pizza-rec}, подставляя в выражение последовательно её саму:

\begin{align*}
L_n &= L_{n-1} + n \\
    &= L_{n-2} + (n-1) + n \\
    &= L_{n-3} + (n-2) + (n-1) + n \\
    & \qquad\vdots \\
    &= L_0 + 1 + 2 + \cdots + (n-2) + (n-1) + n \\
    &= 1 + S_n, \text{ где $S_n$ --- сумма первых $n$ натуральных чисел}
\end{align*}

Последовательность $S_n$ начинается следующими числами: $1, 3, 6, 10, 15, 21, 28, 36, 45, 55, 66, \ldots$.
Эти числа называются также \emph{треугольными числами}, поскольку $S_n$ представляет собой
число кеглей, расставленных треугольником в $n$ рядов.

\subsection{Сумма натуральных чисел}

Метод Гаусса --- складываем числа парами. Первое --- с последним, второе --- с предпоследним,
и так далее:

\begin{equation}
\begin{array}{cccccccccccccc}
  & S_n & = & 1 & + & 2 & + & 3 & + & \cdots & + & (n-1) & + & n \\
+ & S_n & = & n & + & (n-1) & + & (n-2) & + & \cdots & + & 2 & + & 1 \\
\hline
  & 2S_n & = & (n+1) & + & (n+1) & + & (n+1) & + & \cdots & + & (n+1) & + & (n+1)
\end{array}
\end{equation}

Сумма каждой пары таких чисел равна $n+1$, поэтому $S_n = \frac{n(n+1)}{2}$ при $n \ge 0$.

Поэтому $L_n = \frac{n(n+1)}{2} + 1$ при $n \ge 0$.
Это выражение можно проверить с помощью математической индукции:

$$
L_n = L_{n-1} + n = \left(\frac{1}{2}(n-1)n+1 \right)+n = \frac{1}{2}n(n+1) + 1
$$

\section{Число перестановок и факториал}
\section{Числа Фибоначчи}

Числа Фибоначчи определяются следующим рекуррентным соотношением:
\begin{align*}
F_0 &= 0, \\
F_1 &= 1, \\
F_n &= F_{n-1} + F_{n-2}, \text{ для $n>0$}
\end{align*}

Классическое приложение чисел Фибоначчи --- это определение размера популяции кроликов.
Если кролики становятся способными к размножению через месяц и каждая пара взрослых кроликов
производит на свет пару крольчат раз в месяц, то сколько пар кроликов будет в наличии
через $n$ месяцев.

\begin{task}
Написать программу, выводящую первые $n$ чисел Фибоначчи.
\end{task}

\begin{task}
Написать программу для разложения натурального числа на числа Фибоначчи.
\end{task}

\section{Домино}

\subsection{Покрытие прямоугольника $2\times n$}

Как велико число $T_n$ способов покрытия $2\times n$-прямоугольника костяшками домино размера
$2\times 1$? Будем считать, что все костяшки идентичны, следовательно имеет значение только
ориентация костяшки --- вертикальная или горизонтальная: мы можем представлять себе, что
работаем с плитками в форме домино. Например, существует $3$ покрытия $2\times 3$-прямоугольника,
а именно, |||, |= и =|, так что $T_3 = 3$. Для $n=1$ существует одно покрытие |. Для $n=2$
--- два покрытия: || и =. Для $n=0$ покрытие только одно, ведь есть только один
способ не класть ни одной костяшки.

Для $n=4$ имеется две возможности расположения плиток на левом конце прямоугольника
--- можно положить там либо одну плитку вертикально, либо две горизонтально. Если мы выберем
вертикальную плитку, то получим частичное решение |xxx и оставшийся $2\times 3$ прямоугольник
можно покрыть плитками $T_3$ способами. Если же выбрать две горизонтальные плитки, то
частичное решение =xx можно завершить $T_2$ способами. Таким образом, $T_4=T_3+T_2=5$.
Вот эти пять покрытий: ||||, ||=, |=|, =||, ==.

Рассуждения для $n=4$ легко обобщить для любого значения $n \ge 2$: $T_n = T_{n-1} + T_{n-2}$.
Таким образом, получается такое же рекуррентное соотношение, как и для чисел Фибоначчи,
только начальные значения другие. Однако это --- последовательные члены ряда Фибоначчи
$F_1$ и $F_2$, так что последовательность $T$ совпадает с последовательностью Фибоначчи,
сдвинутой на одну позицию: $T_n = F_{n+1} \text{ при $n \ge 0$}$.

\subsection{Покрытие прямоугольника $3\times n$}

Сколько существует способов покрыть костяшками домино $3 \times n$-прямоугольник
(обозначим это число $U_n$)? Пустое покрытие $U_0 = 1$. Для $n=1$ не существует допустимых
покрытий, поскольку одна $2\times 1$-плитка не заполняет $3 \times 1$-прямоугольник,
а для двух нет места.

Следующий случай, $n=2$, легко анализируется вручную: имеется три покрытия ${||}\atop{-}$,
${-}\atop{||}$ и $\equiv$, так что $U_2 = 3$. Для $n=3$ нет покрытий, так как его
площадь нечётна и поэтому ее не получится собрать из костяшек чётной площади.

Каждое непустое покрытие начинается с $\equiv$, L или Г. Но в последних двух случаях
не получится составить рекуррентность для $U_n$, так как правый край у них неровный.
Поэтому можно ввести дополнительную рекуррентность для покрытия. $V_n$  --- число
способов замостить $3\times n$-прямоугольник, оставив левый нижний угол пустым.
Аналогично $\Lambda_n$ --- число
способов замостить $3\times n$-прямоугольник, оставив левый верхний угол пустым.

Входящие в $V_n$ и $\Lambda_n$ покрытия могут начинаться двумя способами: с одной вертикальной
костяшки, которая выравнивает уже заполненный край и с трех костяшек, одна из которых сдвинута вправо.
Тогда,

\begin{align*}
U_0 &= 1 \\
U_1 &= 0 \\
U_n &= U_{n-2} + V_{n-1} + \Lambda_{n-1} \quad & \text{ для $n>1$}\\
V_0 &= 0 \\
V_1 &= 1 \\
V_n &= U_{n-1} + V_{n-2} & \text{ для $n>1$}\\
\Lambda_0 &= 0 \\
\Lambda_1 &= 1 \\
\Lambda_n &= U_{n-1} + \Lambda_{n-2} & \text{ для $n>1$}\\
\end{align*}

Очевидно, что $V_n = \Lambda_n$. Тогда рекуррентное соотношение можно упростить:

\begin{align*}
U_0 &= 1 \\
U_1 &= 0 \\
U_n &= U_{n-2} + 2V_{n-1} \quad & \text{ для $n>1$}\\
V_0 &= 0 \\
V_1 &= 1 \\
V_n &= U_{n-1} + V_{n-2} & \text{ для $n>1$}\\
\end{align*}

\begin{task}
Написать программу, для нахождения числа покрытий $3\times n$-прямоугольника костяшками домино.
\end{task}

\section{Размен}

Сколько существует способов заплатить $50$ центов, если есть монеты по $1$, $5$, $10$, $25$ и $50$ центов.

Используем несколько последовательностей. $P_n$ --- число способов заплатить сумму в $n$ центов
с помощью только одноцентовых монет. Очевидно, что $P_n = 1$ для всех $n>0$. 

Пусть $N_n$ --- число способов заплатить $n$ центов с помощью монет в $5$ и $1$ цент.
Чтобы заплатить эту сумму, нужно или взять одну монету в $5$ центов и выбрать остальное
$N_{n-5}$ способами, либо использовать только одноцентовые монеты и набрать сумму $P_n$
способами.

Аналогичные рассуждения можно провести для остальных монет (ряды $D_n$, $Q_n$, $C_n$). 
Тогда получится следующая рекуррентность
(все выражения для $n \ge 0$):

\begin{align*}
P_n &= 1 \\
N_n &= N_{n-5} + P_n \\
D_n &= D_{n-10} + N_n \\
Q_n &= Q_{n-25} + D_n \\
C_n &= C_{n-50} + Q_n \\
\end{align*}

Ответом на задачу будет $C_n$. Найти его можно построив таблицу:

$$
\begin{array}{c|cccccccccccc}
n   & 0 & 5 & 10 & 15 & 20 & 25 & 30 & 35 & 40 & 45 & 50 \\
\hline
P_n & 1 & 1 & 1  & 1  & 1  & 1  & 1  & 1  & 1  & 1  & 1  \\
N_n & 1 & 2 & 3  & 4  & 5  & 6  & 7  & 8  & 9  & 10 & 11 \\
D_n & 1 & 2 & 4  & 6  & 9  & 12 & 16 &    & 25 &    & 36 \\
Q_n & 1 &   &    &    &    & 13 &    &    &    &    & 49 \\
C_n & 1 &   &    &    &    &    &    &    &    &    & 50 \\
\end{array}
$$

Пустые клетки в таблице могли бы содержать значения, но они не используются при вычислении
$C_{50}$.

\section{Биномиальные коэффициенты}

Символ $\binom{n}{k}$ --- это биномиальный коэффициент. Другое обозначение --- $C_n^k$.
Читается он как <<выбор $k$ из $n$>> или <<$C$ из $n$ по $k$>>. Это число способов выбора
$k$-элементного подмножества из $n$-элементного множества. Например, два элемента
из множества $\{1,2,3,4\}$ можно выбрать шестью способами: 
$\{1,2\}, \{1,3\}, \{1,4\}, \{2,3\}, \{2,4\}, \{3,4\}$, так что $\binom{4}{2}=6$.

Чтобы найти выражение $\binom{n}{k}$, установим соотношение для числа $k$-элементных
последовательностей, выбранных из $n$-элементного множества, в которых учитывается порядок
элементов. Существует $n$ вариантов выбора первого элемента, $n-1$ --- второго элемента
и так далее до $n-k+1$ вариантов выбора $k$ элемента, что дает $n(n-1)\ldots(n-k+1)$
вариантов выбора.

Поскольку каждое $k$-элементное подмножество может быть упорядочено $k!$ различными способами,
найденное число последовательностей учитывает каждое подмножество $k!$ раз. Поэтому чтобы
получить число подмножеств, поделим число перестановок на $k!$:
$$\binom{n}{k} = \frac{n(n-1)\ldots(n-k+1)}{k(k-1)\ldots(1)}$$.

Таблица с биномиальными коэффициентами называется треугольником Паскаля:

$$
 \begin{array}{ccccccccccc}
 &&&&& 1 \\
 &&&& 1 && 1 \\
 &&& 1 && 2 && 1 \\
 && 1 && 3 && 3 && 1 \\
 & 1 && 4 && 6 && 4 && 1 \\
 1 && 5 && 10 && 10 && 5 && 1
 \end{array}
$$

$$
 \begin{array}{ccccccc}
 n & \binom{n}{0} & \binom{n}{1} & \binom{n}{2} & \binom{n}{3} & \binom{n}{4} & \binom{n}{5} \\
 0 & 1 \\
 1 & 1 & 1 \\
 2 &  1 & 2 & 1 \\
 3 & 1 & 3 & 3 & 1 \\
 4 & 1 & 4 & 6 & 4 & 1 \\
 5 & 1 & 5 & 10 & 10 & 5 & 1
 \end{array}
$$

Пустые места в таблице означают $0$ из-за нуля в числителе.

\begin{task}
Напишите программу, рисующую треугольника Паскаля до $n=24$ включительно.
\end{task}

Каждый ряд в треугольнике Паскаля симметричен, потому что $\binom{n}{k}=\binom{n}{n-k}$.
Комбинаторный смысл: определяя $k$ предметов, мы автоматически определяем оставшиеся $n-k$.

Правило внесения/вынесения:
$$\binom{n}{k} = \frac{n}{k}\binom{n-1}{k-1}$$

Формула сложения: каждое число в треугольнике Паскаля есть сумма двух чисел предыдущего ряда ---
того, что непосредственно над ним, и того, что сверху слева от него:
$$\binom{n}{k} = \binom{n-1}{k} + \binom{n-1}{k-1}$$.

Эту формулу можно обосновать с помощью исходного определения:
\begin{align*}
  \binom{n-1}{k} + \binom{n-1}{k-1} 
  &= \frac{(n-1)(n-2)\ldots(n-k)}{k(k-1)\ldots(1)} 
     + \frac{(n-1)(n-2)\ldots(n-k+1)}{(k-1)(k-2)\ldots(1)}  \\
  &= (n-k)\frac{(n-1)(n-2)\ldots(n-k+1)}{k(k-1)\ldots(1)} 
     + k\frac{(n-1)(n-2)\ldots(n-k+1)}{k(k-1)\ldots(1)} = \\
  &= (n-k+k)\frac{(n-1)(n-2)\ldots(n-k+1)}{k(k-1)\ldots(1)} \\
  &= \frac{n(n-1)(n-2)\ldots(n-k+1)}{k(k-1)\ldots(1)} = \binom{n}{k}
\end{align*}

Как обосновать формулу сложения? Предположим, у нас $n$ яиц, среди которых
одно тухлое. Есть $\binom{n}{k}$ способов выбрать $k$ яиц. В $\binom{n-1}{k}$
случаях все они будут свежие, а в $\binom{n-1}{k-1}$ случаях среди них будет
одно тухлое (здесь мы выбираем $k-1$ свежее яйцо и добавляем к ним тухлое).

Формула сложения --- это рекуррентность для чисел из треугольника Паскаля.

\begin{task}
Напишите программу, рисующую треугольника Паскаля до $n=24$ включительно,
используя формулу сложения.
\end{task}

Своим названием биномиальные коэффициенты обязаны \emph{биномиальной теореме}, которая
имеет дело со степенями бинома $x+y$. Вот простейшие случаи этой теоремы:
\begin{align*}
(x+y)^0 &= 1x^0y^0 \\
(x+y)^1 &= 1x^1y^0 + 1x^0y^1 \\
(x+y)^2 &= 1x^2y^0 + 2x^1y^1 + 1x^0y^2 \\
(x+y)^3 &= 1x^3y^0 + 3x^2y^1 + 3x^1y^2 + 1x^0y^3
\end{align*}

Нетрудно понять, почему эти коэффициенты совпадают с числами треугольника Паскаля.
Когда мы раскрываем скобки произведения сомножителей $(x+y)$,
каждый его член сам является произведением $n$ сомножителей $x$ и $y$.
Число таких членом с $k$ сомножителями $x$ и $n-k$ сомножителями $y$ после приведения
подобных членов становится коэффициентом при $x^ky^{n-k}$. А это равно числу
способов выбора $k$ из $n$ двучленов, из которых в произведение войдёт $x$,
т.е. это $\binom{n}{k}$.

Биномиальная теорема:
\begin{equation}
\label{formula-binom}
(x+y)^n = \sum_{k=0}^{n}\binom{n}{k}x^ky^{n-k}
\end{equation}

Если подставить в формулу~\ref{formula-binom} $x=1$ и $y=1$, то получится
\begin{equation}
2^n = \binom{n}{0} + \binom{n}{1} + \binom{n}{2} + \cdots + \binom{n}{n-1} + \binom{n}{n}
\end{equation}

То есть сумма ряда $n$-го ряда треугольника Паскаля равна $2^n$.

\section{Числа Стирлинга второго рода}

Числа Стирлинга второго рода --- число разбиений множества из $n$ элементов на $k$
непустых подмножеств. Обозначаются числа Стирлинга второго рода как $\stirling{n}{k}$.
Так, существует семь способор разбиения четырехэлементного множества на две части:
$$
  \{1,2,3\}\cup\{4\}, \quad \{1,2,4\}\cup\{3\}, \quad \{1,3,4\}\cup\{2\}, \quad \{2,3,4\}\cup\{1\},
  \quad \{1,2\}\cup\{3,4\}, \quad \{1,3\}\cup\{2,4\}, \quad \{1,4\}\cup\{2,3\}
$$

Следовательно, $\stirling{4}{2} = 7$.

Существует только один способ помещения $n$ элементов в одно непустое подмножество.
Следовательно, $\stirling{n}{1} = 1$ при любом $n>0$. С другой стороны, $\stirling{0}{1} = 0$,
потому что $0$-элементное множество пусто. $\stirling{0}{0} = 1$, потому что пустое множество
можно разбить на нулевое число подмножеств только одним способом. А $\stirling{n}{0} = 0$ для
$n>0$, потому что элементы множества нужно поместить хотя бы в одну часть.

Чтобы вычислить $\stirling{n}{k}$, можно использовать следующие рассуждения.
Последний объект из заданных $n$ можно поместить в отдельное подмножество. Остальные можно
разбить на множества $\stirling{n-1}{k-1}$ способами. Либо этот объект можно поместить
в одну из $k$ частей, полученных из $n-1$ объектов. Количество разбиений на такие части
будет равняться $\stirling{n-1}{k}$. Для каждого есть $k$ способов размещения последнего
элемента.
Следовательно,

$$\stirling{n}{k} = k\stirling{n-1}{k} + \stirling{n-1}{k-1}, \text{ для $n>0$}$$

\section{Числа Эйлера}

Число Эйлера $\euler{n}{k}$ --- это число перестановок $\pi_1\pi_2\ldots\pi_n$
множества $\{1,2,\ldots,n\}$, имеющих
$k$ участков подъема, а именно, $k$ мест, где $\pi_j<\pi_{j+1}$.

К примеру, одиннадцать перестановок множества $\{1,2,3,4\}$ содержат по
два участка подъема: $$1324, 1423, 2314, 2413, 3412, 1243, 1342, 2341, 2134, 3124, 4123$$.

Найдём рекуррентность для $\euler{n}{k}$. Каждая перестановка $\pi = \pi_1\pi_2\ldots\pi_{n-1}$
множества $\{1,\ldots,n-1\}$ приводит к $n$ перестановкам множества $\{1,\ldots,n\}$,
если вставлять новый элемент во все возможные места. Предположим, мы вставляем $n$ на $j$-е
место, получая $\pi' = \pi_1\ldots\pi_{j-1}n\pi_j\ldots\pi_{n-1}$. Если $j=1$ или если
$\pi_{j-1} < \pi_j$, то число участков подъема в $\pi'$ такое же, как и в $\pi$. Если же
$\pi_{j-1} > \pi_j$ или же $j=n$, то оно на единицу больше числа участков подъёма в $\pi$.
Поэтому в целом перестановка $\pi'$ с $k$ участками подъёма получается $(k+1)\euler{n-1}{k}$
способами из перестановок $\pi$, которые содержат $k$ участков подъёма, плюс
$((n-2)-(k-1)+1)\euler{n-1}{k-1}$ способами из перестановок $\pi$, которые содержат
$k-1$ участков подъёма.

Итак, искомая рекуррентность:
\begin{align*}
\euler{0}{k} &= [k = 0] \\
\euler{n}{k} &= (k+1)\euler{n-1}{k} + (n-k)\euler{n-1}{k-1}, \text{ для $n>0$} \\
\end{align*}

\section{Как решать задачи с рекуррентностями}

\begin{itemize}
\item Рассмотреть крайние случаи. Они помогают вникнуть в задачу и проверить
      следующие этапы.
\item Найти математическое выражение для интересующей величины ---
      рекуррентность, которая позволяет при заданном $n$ вычислить результат.
\item Найти и доказать замкнутое выражение для рекуррентности.
\end{itemize}



\section{? Производящие функции}
\section{? Свертки}

\chapter{Динамическое программирование}
\section{Одномерное динамическое программирование}
Количество способов решения задачи (задача о кузнечике)

Рассмотрим следующую задачу. На числовой прямой сидит кузнечик, который может прыгать вправо на одну или на две единицы. Первоначально кузнечик находится в точке с координатой 0. Определите количество различных маршрутов кузнечика, приводящих его в точку с координатой n.

Наилучшее решение задачи (задача о кузнечике со стоимостями)

Пусть кузнечик прыгает на одну или две точки вперед, а за прыжок в каждую точку необходимо заплатить определенную стоимость, различную для различных точек. Стоимость прыжка в точку i задается значением Price[i] списка Price. Необходимо найти минимальную стоимость маршрута кузнечика из точки 0 в точку n.

\section{Двумерное динамическое программирование}

Рассмотрим шахматную доску в левом верхнем углу которой находится король. Король может перемещаться только вправо, вниз или по диагонали вправо-вниз на одну клетку. Необходимо определить количество различных маршрутов короля, приводящих его в правый нижний угол.

Теперь решим задачу о нахождении маршрута минимальной стоимости из левого верхнего угла в правый нижний, считая что для каждой клетке указана стоимость прохода через эту клетку.

Игры

Рассмотрим игру «Ферзя в угол» для двух игроков. В левом верхнем углу доски размером $n \times m$ 
находится ферзь, который может двигаться только вправо-вниз. Игроки по очереди двигают ферзя, то есть 
за один ход игрок может переместить ферзя либо по вертикали вниз, либо по горизонтали вправо, либо во 
диагонали вправо-вниз. Игрок, который не может сделать ход — проигрывает, иными словами, выигрывает 
игрок, который поставит ферзя в правый нижний угол. Необходимо определить, какой из игроков может 
выиграть в этой игре независимо от ходов другого игрока.

Наибольшая общая подпоследовательность.

Наибольшая возрастающая подпоследовательность.

В банкомате имеется банкноты nn различных номиналов $a_1, a_2, \ldots, a_n$. Клиент хочет получить сумму в $K$ денежных единиц. Необходимо определить, при помощи какого минимального числа банкнот можно выдать эту сумму (а при необходимости восстановления ответа - определить способ выдачи, использующий минимальное число банкнот).

Есть $n$ золотых слитков массами $a_1, a_2, \ldots, a_n$. Какую максимальную массу золота можно унести, если она не может превышать $K$.

ДИСКРЕТНАЯ ЗАДАЧА ОБ УКЛАДКЕ РЮКЗАКА

Следующим обобщением задачи про золотые слитки является <<задача об укладке рюкзака>>. В этой задаче также имеется несколько предметов, для каждого предмета заданы две характеристики: вес $w_i > 0$ и стоимость предмета $p_i > 0$. Необходимо выбрать множество предметов суммарной максимальной стоимости, при этом суммарная масса выбранных предметов должна быть ограничена значением $K$.

\chapter{Перебор и динамическое программирование}

\chapter{Упражнения}
\begin{enumerate}
\item Некоторые из областей, очерчиваемых $n$ прямыми на плоскости, бесконечны,
в то время как другие конечны. Каково максимально возможное число конечных областей?

\item Один эксцентричный коллекционер покрытий при помощи домино $2 \times n$-прямоугольника
платит 4 доллара за каждую вертикально расположенную костяшку и 1 доллар --- за горизонтальную.
Сколько покрытий будут оценены по этому способу ровно в $m$ долларов? Например, для $m=6$ имеем
три решения: |=, =| и ===.

\item Решите рекуррентное соотношение
$$
\begin{array}{ll}
g_0 = 1; \\
g_n = g_{n-1} + 2g_{n-2} + \cdots + ng_0 & n>0
\end{array}
$$

\item Сколько существует способов разместить числа ${1, 2, \ldots , 2n}$ 
в виде массива размера $2 \times n$ так, чтобы и строки и столбцы массива
были упорядочены по возрастанию слева направо и сверху вниз?
Например, для $n=5$ одним из решений будет 
$$\begin{pmatrix}
1 & 2 & 4 & 5 & 8 \\
3 & 6 & 7 & 9 & 10 
\end{pmatrix}$$.

\item Число Бэлла $\varpi_n$ --- число способов разбиения множества из $n$ предметов на подмножества.
      Например, $\varpi_3=5$, поскольку множество $\{1,2,3\}$ можно разбить на такие подмножества:
      $$\{1,2,3\}; \{1,2\}\cup\{3\}; \{1,3\}\cup\{2\}; \{1\}\cup\{2,3\}; \{1\}\cup\{2\}\cup\{3\}; $$
      Найдите выражение для $\varpi_n$.

\item Грабитель врывается в банк и требует $500$ долларов десяти- и двадцатидолларовыми банкнотами.
      Он также желает знать, сколькими способами кассир может дать ему эти деньги.

\item Сколько существует способов построить $2 \times 2 \times n$-колонну из кирпичей
      размера $2 \times 2 \times 1$?
      
\item Космический зонд обнаружил, что органическое вещество на Марсе имеет ДНК, в состав
      которой входят пять символов, обозначаемых $(a, b, c, d, e)$. Четыре пары символов
      --- $cd$, $ce$, $ed$ и $ee$ --- никогда не встречаются в марсианских ДНК.
      Например, цепочка $bbcda$ запрещена, а $bbdca$ --- разрешена.
      Сколько цепочек ДНК длины $n$ может существовать на Марсе?
      Для $n=2$ ответ $21$, поскольку мы различаем левый и правый концы цепочки.

\item Если $S$ --- некоторое множество целых чисел, то пусть $S+1$ будет <<сдвинутым>>
множеством $\{x+1 \mid x \in S\}$. Сколько подмножеств множества $\{1,2,\ldots,n\}$ обладают
тем свойством, что $S\cup(S+1) = \{1,2,\ldots,n+1\}$.

\end{enumerate}

\section{Ответы}
\begin{enumerate}
\item Если новая $n$-я прямая пересекает прежние прямые в $k>0$ различных точках,
мы получаем $k-1$ новых конечных областей (в предположении, что ни одна из прежних прямых
не параллельна никакой другой) и $2$ новые бесконечные области. Следовательно, максимальное
число конечных областей равно $(n-2) + (n-3) + \cdots = S_{n-2} = (n-1)(n-2)/2 = L_n-2n$.

\item Подставим в производящую функцию $z^4$ вместо | и z вместо -. Получим функцию $1/(1-z^4-z^2)$.
Это похоже на производящую функцию для $T$, однако $z$ заменено на $z^2$. Следовательно, 
ответом будет $0$ для нечетных $m$ и $F_{m/2+1}$ --- для четных.

\item $G(z)=(z/(1-z)^2)G(z)+1$, следовательно 
$$G(z) = \frac{1-2z+z^2}{1-3z+z^2} = 1 + \frac{z}{1-3z+z^2};$$
поэтому имеем $g_n=F_{2n}+[n=0]$.

\item $C_n$. Числа в верхнем ряду соответствуют позициям плюс единиц в последовательности
из $+1$ и $-1$, определяющей <<горную гряду>>; числа нижнего ряда отвечают позициям
минус единиц. Например, приведенный в упражнении массив соответствует
\begin{verbatim}
    /\     
 /\/  \/\  
/        \ 
\end{verbatim}.

\item $\varpi_n = \sum_k C_n^k\varpi_{n-k}$

\item Это задача размена с номиналами монет $10$ и $20$, поэтому $G(z)=(1/(1-z^{10})(1-z^{20})$.

\item Обозначим искомое число $a_n$, а через $b_n$ обозначим разбиение колонны с выемкой размера
      $2 \times 2 \times 1$ вверху. Рассматривая различные варианты расположения видимых сверху
      кирпичей, получим
$$
\begin{array}{ll}
a_n = 2a_{n-1} + 4b_{n-1} + a_{n-2} + [n=0]; \\
b_n = a_{n-1} + b_{n-1}.
\end{array}
$$
      Следовательно, производящие функции удовлетворяют уравнениям $A = 2zA+4zB+z^2A+1$ и
      $B = zA + zB$. Откуда находим $$A(z) = \frac{1-z}{(1+z)(1-4z+z^2)}$$
      $$a_n = \frac{1}{6}(2+\sqrt{3})^{n+1} + \frac{1}{6}(2-\sqrt{3})^{n+1} + \frac{1}{3}(-1)^n$$

\item Пусть $a_n$ --- число цепочек ДНК, которые не заканчиваются на $c$ или $e$, а $b_n$ --- число
      цепочек, заканчивающихся этими символами. Тогда
      $$
      \begin{array}{ll}
      a_n = 3a_{n-1}+2b_{n-1}+[n=0], & b_n = 2a_{n-1} + b_{n-1}, \\
      A(z) = 3zA(z) + 2zB(z) + 1,    & B(z) = 2zA(z) + zB(z), \\
      A(z) = \frac{1-z}{1-4z-z^2},   & B(z) = \frac{2z}{1-4z-z^2}. \\
      \end{array}
      $$
      Общее число цепочек равно $[z^n](1+z)/(1-4z-z^2) = F_{3n+2}$.

\item $F_n$.

\end{enumerate}

\end{document} 
