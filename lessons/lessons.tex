\documentclass[14pt]{book}

\usepackage[
  a4paper, mag=1000,
  left=2cm, right=1cm, top=2cm, bottom=2cm, headsep=0.7cm, footskip=1.27cm
]{geometry}

\usepackage[T2A]{fontenc}
\usepackage[utf8]{inputenc}
\usepackage[english,russian]{babel}
\usepackage{cmap}
\usepackage{pscyr} % Нормальные шрифты
\usepackage{amsmath}
\usepackage{tabularx}
\usepackage{array}
\usepackage{multirow}
\usepackage{amssymb}

% Форматирование чисел
\usepackage{siunitx}
\sisetup{range-phrase = \ldots}

% оглавление + ссылки
\usepackage[pdftex]{hyperref}

\hypersetup{%
    pdfborder = {0 0 0},
    colorlinks,
    citecolor=red,
    filecolor=Darkgreen,
    linkcolor=blue,
    urlcolor=blue
}

\usepackage{hypcap}
\usepackage[numbered]{bookmark} % для цифр в закладках в pdf viewer'ов

\usepackage{tocloft}

\renewcommand{\cftsecaftersnum}{.}
\renewcommand{\cftsubsecaftersnum}{.}

\setcounter{tocdepth}{4}
\setcounter{secnumdepth}{4}

\renewcommand\cftsecleader{\cftdotfill{\cftdotsep}}
\renewcommand{\cfttoctitlefont}{\hfill\Large\bfseries}
\renewcommand{\cftaftertoctitle}{\hfill}


% листинги
\usepackage{listings}
\usepackage{color}
\definecolor{commentGreen}{rgb}{0,0.6,0}
\usepackage{caption} % для подписей под листингами и таблиц
%\usepackage[scaled]{beramono}

% графика
\usepackage[pdftex]{graphicx}
\graphicspath{{./fig/}} % папка с изображениями

%\usepackage{setspace} % для полуторного интервала
%\onehalfspacing % сам полуторный интервал

\usepackage{indentfirst} % отступ в первом абзаце

% точки в секция и подсекциях
\usepackage{secdot}
\sectiondot{subsection}

% Подавление висячих строк
\clubpenalty=10000
\widowpenalty=10000

% колонтитулы
%\usepackage{fancybox,fancyhdr}
%\pagestyle{fancy}
%\fancyhf{}
%\fancyhead[C]{\small{C++ and Qt}} % шапка верхнего колонтитула!!!
%\fancyfoot[C]{\small{\thepage}} % шапка нижнего колонтитула!!!

% стили листингов кодов

\lstdefinestyle{CPlusPlus}{
  language=C++,
  basicstyle=\small\ttfamily,
  breakatwhitespace=true,
  breaklines=true,
  showstringspaces=false,
  keywordstyle=\color{blue}\ttfamily,
  stringstyle=\color{red}\ttfamily,
  commentstyle=\color{commentGreen}\ttfamily,
  morecomment=[l][\color{magenta}]{\#},
  numbers=left,
  xleftmargin=1cm
}

\begin{document}

\begingroup
\hypersetup{linkcolor=black}
\tableofcontents
\endgroup

\clearpage

\chapter{Рекуррентные соотношения}

$C_n$, $F_n$

\section{Числа Фибоначчи}
\section{Домино}
\section{Размен}
\section{Решение рекуррентных соотношений}
\section{Производящие функции}
\section{Свертки}
\chapter{Динамическое программирование}
\chapter{Перебор и динамическое программирование}

\chapter{Упражнения}
\begin{enumerate}
\item Один эксцентричный коллекционер покрытий при помощи домино $2 \times n$-прямоугольника
платит 4 доллара за каждую вертикально расположенную костяшку и 1 доллар --- за горизонтальную.
Сколько покрытий будут оценены по этому способу ровно в $m$ долларов? Например, для $m=6$ имеем
три решения: |=, =| и ===.

\item Решите рекуррентное соотношение
$$
\begin{array}{ll}
g_0 = 1; \\
g_n = g_{n-1} + 2g_{n-2} + \cdots + ng_0 & n>0
\end{array}
$$

\item Сколько существует способов разместить числа ${1, 2, \ldots , 2n}$ 
в виде массива размера $2 \times n$ так, чтобы и строки и столбцы массива
были упорядочены по возрастанию слева направо и сверху вниз?
Например, для $n=5$ одним из решений будет 
$$\begin{pmatrix}
1 & 2 & 4 & 5 & 8 \\
3 & 6 & 7 & 9 & 10 
\end{pmatrix}$$.
\end{enumerate}

\section{Ответы}
\begin{enumerate}
\item Подставим в производящую функцию $z^4$ вместо | и z вместо -. Получим функцию $1/(1-z^4-z^2)$.
Это похоже на производящую функцию для $T$, однако $z$ заменено на $z^2$. Следовательно, 
ответом будет $0$ для нечетных $m$ и $F_{m/2+1}$ --- для четных.

\item $G(z)=(z/(1-z)^2)G(z)+1$, следовательно 
$$G(z) = \frac{1-2z+z^2}{1-3z+z^2} = 1 + \frac{z}{1-3z+z^2};$$
поэтому имеем $g_n=F_{2n}+[n=0]$.

\item $C_n$. Числа в верхнем ряду соответствуют позициям плюс единиц в последовательности
из $+1$ и $-1$, определяющей <<горную гряду>>; числа нижнего ряда отвечают позициям
минус единиц. Например, приведенный в упражнении массив соответствует
\begin{verbatim}
    /\     
 /\/  \/\  
/        \ 
\end{verbatim}.

\end{enumerate}

\end{document} 
