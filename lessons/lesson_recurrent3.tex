\section{Размен}

Сколько существует способов заплатить $50$ центов, если есть монеты по $1$, $5$, $10$, $25$ и $50$ центов.

Используем несколько последовательностей. $P_n$ --- число способов заплатить сумму в $n$ центов
с помощью только одноцентовых монет. Очевидно, что $P_n = 1$ для всех $n>0$. 

Пусть $N_n$ --- число способов заплатить $n$ центов с помощью монет в $5$ и $1$ цент.
Чтобы заплатить эту сумму, нужно или взять одну монету в $5$ центов и выбрать остальное
$N_{n-5}$ способами, либо использовать только одноцентовые монеты и набрать сумму $P_n$
способами.

Аналогичные рассуждения можно провести для остальных монет (ряды $D_n$, $Q_n$, $C_n$). 
Тогда получится следующая рекуррентность
(все выражения для $n \ge 0$):

\begin{align*}
P_n &= 1 \\
N_n &= N_{n-5} + P_n \\
D_n &= D_{n-10} + N_n \\
Q_n &= Q_{n-25} + D_n \\
C_n &= C_{n-50} + Q_n \\
\end{align*}

TODO: проблема с рекурсией

\begin{task}
Написать рекурсивные функции с кэшированием результата для вычисления рекуррентности.
\end{task}

Ответом на задачу будет $C_n$. Найти его можно построив таблицу:

$$
\begin{array}{c|cccccccccccc}
n   & 0 & 5 & 10 & 15 & 20 & 25 & 30 & 35 & 40 & 45 & 50 \\
\hline
P_n & 1 & 1 & 1  & 1  & 1  & 1  & 1  & 1  & 1  & 1  & 1  \\
N_n & 1 & 2 & 3  & 4  & 5  & 6  & 7  & 8  & 9  & 10 & 11 \\
D_n & 1 & 2 & 4  & 6  & 9  & 12 & 16 &    & 25 &    & 36 \\
Q_n & 1 &   &    &    &    & 13 &    &    &    &    & 49 \\
C_n & 1 &   &    &    &    &    &    &    &    &    & 50 \\
\end{array}
$$

Пустые клетки в таблице могли бы содержать значения, но они не используются при вычислении
$C_{50}$.

\begin{task}
Написать программу, заполняющую таблицу (или набор одномерных массивов) без использования
рекурсии.
\end{task}

\section{Биномиальные коэффициенты}

Символ $\binom{n}{k}$ --- это биномиальный коэффициент. Другое обозначение --- $C_n^k$.
Читается он как <<выбор $k$ из $n$>> или <<$C$ из $n$ по $k$>>. Это число способов выбора
$k$-элементного подмножества из $n$-элементного множества. Например, два элемента
из множества $\{1,2,3,4\}$ можно выбрать шестью способами: 
$\{1,2\}, \{1,3\}, \{1,4\}, \{2,3\}, \{2,4\}, \{3,4\}$, так что $\binom{4}{2}=6$.

Чтобы найти выражение $\binom{n}{k}$, установим соотношение для числа $k$-элементных
последовательностей, выбранных из $n$-элементного множества, в которых учитывается порядок
элементов. Существует $n$ вариантов выбора первого элемента, $n-1$ --- второго элемента
и так далее до $n-k+1$ вариантов выбора $k$ элемента, что дает $n(n-1)\ldots(n-k+1)$
вариантов выбора.

Поскольку каждое $k$-элементное подмножество может быть упорядочено $k!$ различными способами,
найденное число последовательностей учитывает каждое подмножество $k!$ раз. Поэтому чтобы
получить число подмножеств, поделим число перестановок на $k!$:
$$\binom{n}{k} = \frac{n(n-1)\ldots(n-k+1)}{k(k-1)\ldots(1)}$$.

Таблица с биномиальными коэффициентами называется треугольником Паскаля:

$$
 \begin{array}{ccccccccccc}
 &&&&& 1 \\
 &&&& 1 && 1 \\
 &&& 1 && 2 && 1 \\
 && 1 && 3 && 3 && 1 \\
 & 1 && 4 && 6 && 4 && 1 \\
 1 && 5 && 10 && 10 && 5 && 1
 \end{array}
$$

$$
 \begin{array}{ccccccc}
 n & \binom{n}{0} & \binom{n}{1} & \binom{n}{2} & \binom{n}{3} & \binom{n}{4} & \binom{n}{5} \\
 0 & 1 \\
 1 & 1 & 1 \\
 2 &  1 & 2 & 1 \\
 3 & 1 & 3 & 3 & 1 \\
 4 & 1 & 4 & 6 & 4 & 1 \\
 5 & 1 & 5 & 10 & 10 & 5 & 1
 \end{array}
$$

Пустые места в таблице означают $0$ из-за нуля в числителе.

\begin{task}
Напишите программу, рисующую треугольника Паскаля до $n=24$ включительно.
\end{task}

Каждый ряд в треугольнике Паскаля симметричен, потому что $\binom{n}{k}=\binom{n}{n-k}$.
Комбинаторный смысл: определяя $k$ предметов, мы автоматически определяем оставшиеся $n-k$.

Правило внесения/вынесения:
$$\binom{n}{k} = \frac{n}{k}\binom{n-1}{k-1}$$

Формула сложения: каждое число в треугольнике Паскаля есть сумма двух чисел предыдущего ряда ---
того, что непосредственно над ним, и того, что сверху слева от него:
\begin{equation}
\label{formula-binom-rec}
\binom{n}{k} = \binom{n-1}{k} + \binom{n-1}{k-1}
\end{equation}

Эту формулу можно обосновать с помощью исходного определения:
\begin{align*}
  \binom{n-1}{k} + \binom{n-1}{k-1} 
  &= \frac{(n-1)(n-2)\ldots(n-k)}{k(k-1)\ldots(1)} 
     + \frac{(n-1)(n-2)\ldots(n-k+1)}{(k-1)(k-2)\ldots(1)}  \\
  &= (n-k)\frac{(n-1)(n-2)\ldots(n-k+1)}{k(k-1)\ldots(1)} 
     + k\frac{(n-1)(n-2)\ldots(n-k+1)}{k(k-1)\ldots(1)} = \\
  &= (n-k+k)\frac{(n-1)(n-2)\ldots(n-k+1)}{k(k-1)\ldots(1)} \\
  &= \frac{n(n-1)(n-2)\ldots(n-k+1)}{k(k-1)\ldots(1)} = \binom{n}{k}
\end{align*}

Как обосновать формулу сложения? Предположим, у нас $n$ яиц, среди которых
одно тухлое. Есть $\binom{n}{k}$ способов выбрать $k$ яиц. В $\binom{n-1}{k}$
случаях все они будут свежие, а в $\binom{n-1}{k-1}$ случаях среди них будет
одно тухлое (здесь мы выбираем $k-1$ свежее яйцо и добавляем к ним тухлое).

Формула сложения --- это рекуррентность для чисел из треугольника Паскаля.

\begin{task}
Напишите программу, рисующую треугольника Паскаля до $n=24$ включительно,
используя формулу сложения.
\end{task}

Своим названием биномиальные коэффициенты обязаны \emph{биномиальной теореме}, которая
имеет дело со степенями бинома $x+y$. Вот простейшие случаи этой теоремы:
\begin{align*}
(x+y)^0 &= 1x^0y^0 \\
(x+y)^1 &= 1x^1y^0 + 1x^0y^1 \\
(x+y)^2 &= 1x^2y^0 + 2x^1y^1 + 1x^0y^2 \\
(x+y)^3 &= 1x^3y^0 + 3x^2y^1 + 3x^1y^2 + 1x^0y^3
\end{align*}

Нетрудно понять, почему эти коэффициенты совпадают с числами треугольника Паскаля.
Когда мы раскрываем скобки произведения сомножителей $(x+y)$,
каждый его член сам является произведением $n$ сомножителей $x$ и $y$.
Число таких членом с $k$ сомножителями $x$ и $n-k$ сомножителями $y$ после приведения
подобных членов становится коэффициентом при $x^ky^{n-k}$. А это равно числу
способов выбора $k$ из $n$ двучленов, из которых в произведение войдёт $x$,
т.е. это $\binom{n}{k}$.

Биномиальная теорема:
\begin{equation}
\label{formula-binom}
(x+y)^n = \sum_{k=0}^{n}\binom{n}{k}x^ky^{n-k}
\end{equation}

Если подставить в формулу~\ref{formula-binom} $x=1$ и $y=1$, то получится
\begin{equation}
2^n = \binom{n}{0} + \binom{n}{1} + \binom{n}{2} + \cdots + \binom{n}{n-1} + \binom{n}{n}
\end{equation}

То есть сумма ряда $n$-го ряда треугольника Паскаля равна $2^n$.
