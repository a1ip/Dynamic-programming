\begin{enumerate}
\item Один эксцентричный коллекционер покрытий при помощи домино $2 \times n$-прямоугольника
платит 4 доллара за каждую вертикально расположенную костяшку и 1 доллар --- за горизонтальную.
Сколько покрытий будут оценены по этому способу ровно в $m$ долларов? Например, для $m=6$ имеем
три решения: |=, =| и ===.

\item Грабитель врывается в банк и требует $500$ долларов десяти- и двадцатидолларовыми банкнотами.
      Он также желает знать, сколькими способами кассир может дать ему эти деньги.

\item Сколько существует способов построить $2 \times 2 \times n$-колонну из кирпичей
      размера $2 \times 2 \times 1$?
      
\item Космический зонд обнаружил, что органическое вещество на Марсе имеет ДНК, в состав
      которой входят пять символов, обозначаемых $(a, b, c, d, e)$. Четыре пары символов
      --- $cd$, $ce$, $ed$ и $ee$ --- никогда не встречаются в марсианских ДНК.
      Например, цепочка $bbcda$ запрещена, а $bbdca$ --- разрешена.
      Сколько цепочек ДНК длины $n$ может существовать на Марсе?
      Для $n=2$ ответ $21$, поскольку мы различаем левый и правый концы цепочки.


\end{enumerate}
