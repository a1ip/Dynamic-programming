Динамическое программирование --- это технология для эффективной реализации рекурсивных алгоритмов
посредством сохранения промежуточных результатов. Секрет её применения заключается
в определении, выдаёт ли простой рекурсивный алгоритм одинаковые результаты для одинаковых
подзадач. Если выдаёт, то вместо повторения вычислений ответ каждой подзадачи можно
сохранять в таблице для использования в дальнейшем, что даёт возможность получить эффективный
алгоритм. Начинаем разработку с определения и отладки рекурсивного алгоритма. Только добившись
правильной работы нашего рекурсивного алгоритма, мы переходим к поиску мер по ускорению
его работы, сохраняя результаты в памяти.

Многократное вычисление некоего значения безвредно само по себе, пока затраченное на это время
не оказывает заметного влияния на производительность. Экономия времени исполнения за счёт повышенного
расхода памяти в динамическом программировании лучше всего проявляется в при вычислении
рекуррентных соотношений, таких как числа Фибоначчи.

Сохранение результатов вычислений в таблице имеет смысл только если в алгоритме присутствуют
повторяющиеся вычисления. 

Решение задачи с помощью динамического программирования состоит из следующих шагов:
\begin{enumerate}
\item Сформулировать решение в виде рекуррентного соотношения или рекурсивного алгоритма.
\item Показать, что количество разных значений параметра, принимаемых рекуррентностью,
      ограничено полиномиальной функцией.
\item Доказать оптимальность для поздадач.
\item Показать, что есть повторяющиеся подзадачи, позволяющие ускорить вычисления.
\item Указать порядок вычисления рекуррентного соотношения, чтобы частичные результаты
      были доступными, когда они потребуются.
\item Написать программу.
\end{enumerate}

\section{Множество решаемых задач}

Оптимальность для подзадач: шахматы, путь в графе

Перекрывающиеся подзадачи

\section{Граф зависимостей задач}

\section{Пример решения задачи}
