\section{Ханойские башни}

Ханойская башня --- несколько дисков, нанизанных в порядке уменьшения размеров на один из
трех колышков. Задача состоит в том, чтобы переместить всю башню на один из других колышков,
перенося каждый раз только один диск и не помещая больший диск на меньший.
Какое количество перемещений дисков необходимо и достаточно для выполнения задачи?

\textsc{Обозначай и властвуй}: путь $T_n$ --- минимальное число перекладываний, необходимых
для перемещения башни размером $n$ с одного колышка на другой. Очевидно, что $T_1=1$, а $T_2=3$.
Специальный крайний случай --- $T_0=0$, ведь для перемещения башни без дисков вообще ничего делать
не нужно.

Как переместить всю бащню? Эксперементы с тремя дисками показывают, что для такой башни нужно 
сначала переместить два верхних диска на один колышек, потом нижний диск на другой. И уже 
после этого два верхних диска снова перемещаются на самый большой. В общем случае мы сначала
перемещаем $n-1$ меньших дисков на любой из колышков (что требует $T_{n-1}$ перекладываний),
затем перемещаем самый большой диск (одно перекладывание) и, наконец, помещаем $n-1$
меньших дисков обратно на самый большой диск (ещё $T_{n-1}$ перекладываний). 

Таким образом, $n$ дисков (при $n>0$) можно переместить самое большее за $2T_{n-1}+1$
перекладываний. Меньше перекладываний сделать не получится, потому что когда мы хотим
переместить самый большой диск, $n-1$ меньших дисков должны находиться на одном колышке,
а для этого потребуется не меньше $T_{n-1}$ перекладываний.
Несколько раз перекладывать большой диск смысла нет. После его перемещения меньшие диски нужно
переместить на него обратно за $T_{n-1}$ перекладываний. Следовательно,

\begin{equation}
\label{formula-hanoi-rec}
\begin{array}{ll}
T_0 = 0, \\
T_n = 2T_{n-1} + 1 & \text{при } n > 0 \\
\end{array}
\end{equation}

Совокупность равенств типа~\ref{formula-hanoi-rec} называется \emph{рекуррентностью}. Она задается
начальным значением и зависимостью общего члена от предыдущих.

Рекуррентность вида~\ref{formula-hanoi-rec} позволяет вычислять $T_n$ для любого $n$, какое мы пожелаем. Но в действительности
никто не захочет пользоваться для вычисления рекуррентностью, когда $n$ велико --- это займёт
слишком много времени. Рекуррентность даёт только косвенную, локальную информацию.
Решение рекуррентного соотношения --- это запись $T_n$ в простой и компактной
<<замкнутой форме>>. Такая запись позволяет вычислить $T_n$ быстро даже при большом $n$.

Прибавим $1$ к обеим частям уравнения:

\begin{equation}
\begin{array}{ll}
T_0 + 1 = 1, \\
T_n + 1 = 2T_{n-1} + 2 & \text{при } n > 0 \\
\end{array}
\end{equation}

Теперь, если положить $U_n = T_n + 1$, то получим 

\begin{equation}
\begin{array}{ll}
U_0 = 1, \\
U_n = 2U_{n-1} & \text{при } n > 0 \\
\end{array}
\end{equation}

Очевидно, что $U_n = 2^n$. Тогда $T_n = 2^n - 1$.


\subsection{Метод математической индукции}

Значения $T_n$ можно вычислять для разных $n$. Получится $T_0=0, T_1=1, T_2=3, T_3=7, T_4=15$.
Можно предположить, что $T_n=2^n-1$. Проверим это с помощью математической индукции.

База индукции тривиальна --- $T_0 = 0$. Индуктивный переход: если $T_{n-1}=2^{n-1}-1$, то
$T_n = 2T_{n-1}+1 = 2(2^{n-1}-1)+1 = 2^n - 1$. Следовательно, предположение было справедливо.

\section{Как решать задачи с рекуррентностями}

\begin{itemize}
\item Рассмотреть крайние случаи. Они помогают вникнуть в задачу и проверить
      следующие этапы.
\item Найти математическое выражение для интересующей величины ---
      рекуррентность, которая позволяет при заданном $n$ вычислить результат.
\item Найти и доказать замкнутое выражение для рекуррентности.
\end{itemize}

\section{Задача о разрезании пиццы}

Сколько кусков пиццы можно получить, делая $n$ прямолинейных разрезов ножом?
Или же, каково максимальное число $L_n$ областей, на которые плоскость делится
$n$ прямыми?

Рассмотрим крайние случаи. Плоскость без прямых --- это одна область, 
с одной прямой --- две области, а с двумя прямыми --- четыре области.
Может показаться, что $L_n = 2^n$, но в общем случае это не так.
При добавлении третьей прямой мы можем пересечь самое большее три области,
поэтому $L_3 = 7$.

Новая $n$-я прямая увеличивает число областей на $k$, когда рассекает $k$ старых
областей. А рассекает она $k$ старых областей, когда пересекает прежние прямые
в $k-1$ различных местах. Две прямые могут пересекаться не более чем в одной точке.
Поэтому новая прямая может пересекать $n-1$ старых прямых не более чем в $n-1$
различных точках. Если проводить прямую так, чтобы она не была параллельна ни одной
из уже имеющихся, она обязательно пересечет их все. При этом, надо выбирать прямую так,
чтобы она не проходила через уже имеющиеся пересечения. Поэтому искомое рекуррентное
соотношение имеет вид:

\begin{equation}
\label{formula-pizza-rec}
\begin{array}{ll}
L_0 = 1, \\
L_n = L_{n-1} + n & \text{при } n > 0 \\
\end{array}
\end{equation}

Теперь можно вычислить значения этого ряда: $1, 2, 4, 7, 11, 16, \ldots$. 
Но теперь довольно сложно угадать что за замкнутая форма стоит за ними.

\subsection{Разворачивание рекуррентности}

Развернем формулу~\ref{formula-pizza-rec}, подставляя в выражение последовательно её саму:

\begin{align*}
L_n &= L_{n-1} + n \\
    &= L_{n-2} + (n-1) + n \\
    &= L_{n-3} + (n-2) + (n-1) + n \\
    & \qquad\vdots \\
    &= L_0 + 1 + 2 + \cdots + (n-2) + (n-1) + n \\
    &= 1 + S_n, \text{ где $S_n$ --- сумма первых $n$ натуральных чисел}
\end{align*}

Последовательность $S_n$ начинается следующими числами: $1, 3, 6, 10, 15, 21, 28, 36, 45, 55, 66, \ldots$.
Эти числа называются также \emph{треугольными числами}, поскольку $S_n$ представляет собой
число кеглей, расставленных треугольником в $n$ рядов.

\subsection{Сумма натуральных чисел}

Метод Гаусса --- складываем числа парами. Первое --- с последним, второе --- с предпоследним,
и так далее:

\begin{equation}
\begin{array}{cccccccccccccc}
  & S_n & = & 1 & + & 2 & + & 3 & + & \cdots & + & (n-1) & + & n \\
+ & S_n & = & n & + & (n-1) & + & (n-2) & + & \cdots & + & 2 & + & 1 \\
\hline
  & 2S_n & = & (n+1) & + & (n+1) & + & (n+1) & + & \cdots & + & (n+1) & + & (n+1)
\end{array}
\end{equation}

Сумма каждой пары таких чисел равна $n+1$, поэтому $S_n = \frac{n(n+1)}{2}$ при $n \ge 0$.

Поэтому $L_n = \frac{n(n+1)}{2} + 1$ при $n \ge 0$.
Это выражение можно проверить с помощью математической индукции:

$$
L_n = L_{n-1} + n = \left(\frac{1}{2}(n-1)n+1 \right)+n = \frac{1}{2}n(n+1) + 1
$$

\subsection{Задача Иосифа Флавия}

Задача Иосифа Флавия. $n$ человек выстраиваются по кругу и нумеруются числами от $1$ до $n$. Затем из них исключается каждый второй до тех пор, пока не останется только один человек. Например, если $n = 10$, то порядок исключения таков: $2, 4, 6, 8, 10, 3, 7, 1, 9$, так что остается номер $5$. Для данного $n$ будем обозначать через $J(n)$ номер последнего оставшегося человека. Докажите, что 
  \begin{enumerate}
    \item $J(2n) = 2J(n) - 1$
    \item $J(2n + 1) = 2J(n) + 1$
    \item если $n = (1b_{m - 1}b_{m - 2}\ldots b_1b_0)_2$, то $J(n) = (b_{m - 1}b_{m - 2}\ldots b_1b_01)_2$. 

  \end{enumerate}

Если по кругу стоят числа $1, 2, \ldots, 2n$, то вначале вычеркиваются все четные числа. 
Оставшиеся числа $1, 3, 5, \ldots, 2n - 1$ снова подвергаются процедуре вычеркивания. $k$-ое число в этом 
списке имеет вид $2k - 1$. После того, как из этого списка будут вычеркнуты все числа кроме одного, 
останется число с номером $J(n)$, которое равно $2J(n) - 1$. 

