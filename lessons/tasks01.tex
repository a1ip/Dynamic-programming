\begin{enumerate}
\item Некоторые из областей, очерчиваемых $n$ прямыми на плоскости, бесконечны,
в то время как другие конечны. Каково максимально возможное число конечных областей?

% http://www.problems.ru/view_problem_details_new.php?id=60562
\item Некоторый алфавит состоит из 6 букв, которые для передачи по телеграфу кодированы так:
$. - .. -- .- -.$. При передаче одного слова не сделали промежутков, отделяющих букву от буквы, так что получилась сплошная цепочка из точек и тире, содержащая 12 знаков. Сколькими способами можно прочитать переданное слово? 

% http://www.problems.ru/view_problem_details_new.php?id=60561
\item О том, как прыгают кузнечики. Предположим, что имеется лента, разбитая на клетки и уходящая вправо до бесконечности. На первой клетке этой ленты сидит кузнечик. Из любой клетки кузнечик может перепрыгнуть либо на одну, либо на две клетки вправо. Сколькими способами кузнечик может добраться до $n$-ой от начала ленты клетки? 

% http://www.problems.ru/view_problem_details_new.php?id=60583
\item Сколько существует последовательностей из $1$ и $2$, таких что сумма чисел в каждой такой последовательности равна $n$? Например, если $n = 4$, то таких последовательностей пять:
$1111$, $112$, $121$, $211$, $22$.

% http://www.problems.ru/view_problem_details_new.php?id=116925
\item На доске записаны в ряд сто чисел, отличных от нуля. Известно, что каждое число, кроме первого и последнего, является произведением двух соседних с ним чисел. Первое число –-- это $7$. Какое число последнее?

% http://www.problems.ru/view_problem_details_new.php?id=60289
\item Числа $a_0, a_1, \ldots , a_n, \ldots$ определены следующим образом:
$a_0 = 2, a_1 = 3,  a_{n + 1} = 3a_n - 2a_{n - 1} (n \ge 2)$.
Найдите и докажите формулу для этих чисел. 

% http://www.problems.ru/view_problem_details_new.php?id=104123
\item \begin{enumerate}
    \item Леша поднимается по лестнице из $10$ ступенек. За один раз он прыгает вверх либо на одну  
          ступеньку, либо на две ступеньки. Сколькими способами Леша может подняться по лестнице? 
    \item При спуске с той же лестницы Леша перепрыгивает через некоторые ступеньки (может даже 
          через все $10$). Сколькими способами он может спуститься по этой лестнице?
\end{enumerate}

% http://www.problems.ru/view_problem_details_new.php?id=35468
\item Найдите количество слов длины 10, состоящих только из букв <<а>> и <<б>> и не содержащих в записи двух букв <<б>> подряд. 

% http://www.problems.ru/view_problem_details_new.php?id=116880
\item Последовательность an задана условием: $a_{n+1} = a_n – a_{n–1}$.
      Найдите $a_{100}$, если $a_1 = 3,  a_2 = 7$.

% http://www.problems.ru/view_problem_details_new.php?id=78286
\item Доказать, что любое натуральное число можно представить в виде суммы нескольких различных членов последовательности $1, 2, 3, 5, 8, 13, \ldots, a_n = a_{n - 1} + a_{n - 2}, \ldots$. 

% http://www.problems.ru/view_problem_details_new.php?id=61462
\item Найдите формулу $n$-го члена для последовательностей, заданных условиями ($n \ge 0$): 
  \begin{enumerate}
    \item $a_0 = 0, a_1 = 1, a_{n + 2} = 5a_{n + 1} - 6a_n$
    \item $a_0 = 1, a_1 = 1, a_{n + 2} = 3a_{n + 1} - 2a_n$
    \item $a_0 = 1, a_1 = 1, a_{n + 2} = a_{n + 1} + a_n$
    \item $a_0 = 1, a_1 = 2, a_{n + 2} = 2a_{n + 1} - a_n$
    \item $a_0 = 0, a_1 = 1, a_{n + 2} = 2a_{n + 1} + a_n$
  \end{enumerate}

\item Задача Иосифа Флавия. $n$ человек выстраиваются по кругу и нумеруются числами от $1$ до $n$. Затем из них исключается каждый второй до тех пор, пока не останется только один человек. Например, если $n = 10$, то порядок исключения таков: $2, 4, 6, 8, 10, 3, 7, 1, 9$, так что остается номер $5$. Для данного $n$ будем обозначать через $J(n)$ номер последнего оставшегося человека. Докажите, что 
  \begin{enumerate}
    \item $J(2n) = 2J(n) - 1$
    \item $J(2n + 1) = 2J(n) + 1$
    \item если $n = (1b_{m - 1}b_{m - 2}\ldots b_1b_0)_2$, то $J(n) = (b_{m - 1}b_{m - 2}\ldots b_1b_01)_2$. 

  \end{enumerate}

% http://www.problems.ru/view_problem_details_new.php?id=61476
\item Садовник, привив черенок редкого растения, оставляет его расти два года, а затем ежегодно берет от него по $6$ черенков. С каждым новым черенком он поступает аналогично. Сколько будет растений и черенков на $n$-ом году роста первоначального растения? 

% http://www.problems.ru/view_problem_details_new.php?id=61474
\item Лягушка прыгает по вершинам шестиугольника $ABCDEF$, каждый раз перемещаясь в одну из соседних вершин. Сколькими способами она может попасть из $A$ в $C$ за $n$ прыжков? Тот же вопрос, но при условии, что ей нельзя прыгать в $D$? 

% http://www.problems.ru/view_problem_details_new.php?id=61473
\item Лягушка прыгает по вершинам треугольника $ABC$, перемещаясь каждый раз в одну из соседних вершин. Сколькими способами она может попасть из $A$ в $A$ за $n$ прыжков?

% http://www.problems.ru/view_problem_details_new.php?id=65777
\item Высокий прямоугольник ширины $2$ открыт сверху, и в него падают в разной ориентации Г-тримино.
  Сколькими способами можно получить пирамиду высоты $n$?

% http://www.problems.ru/view_problem_details_new.php?id=65275
\item Трасса для картинга состоит из кольца и ответвления из точки $B$ к точке старта/финиша $A$,
 причем картингист по дороге может сколько угодно раз заезжать в точку $A$ и возвращаться на круг.
 На путь от $A$ до кольца (или обратно) юный гонщик Юра тратит минуту. На путь по кольцу он также
 тратит минуту. По кольцу можно ездить только против часовой стрелки. Юра не поворачивает назад на полпути и не останавливается. Длительность заезда $10$ минут. Найдите число возможных различных маршрутов (последовательностей прохождения участков).


\end{enumerate}
