\section{Домино}

\subsection{Покрытие прямоугольника $2\times n$}

Как велико число $T_n$ способов покрытия $2\times n$-прямоугольника костяшками домино размера
$2\times 1$? Будем считать, что все костяшки идентичны, следовательно имеет значение только
ориентация костяшки --- вертикальная или горизонтальная: мы можем представлять себе, что
работаем с плитками в форме домино. Например, существует $3$ покрытия $2\times 3$-прямоугольника,
а именно, |||, |= и =|, так что $T_3 = 3$. Для $n=1$ существует одно покрытие |. Для $n=2$
--- два покрытия: || и =. Для $n=0$ покрытие только одно, ведь есть только один
способ не класть ни одной костяшки.

Для $n=4$ имеется две возможности расположения плиток на левом конце прямоугольника
--- можно положить там либо одну плитку вертикально, либо две горизонтально. Если мы выберем
вертикальную плитку, то получим частичное решение |xxx и оставшийся $2\times 3$ прямоугольник
можно покрыть плитками $T_3$ способами. Если же выбрать две горизонтальные плитки, то
частичное решение =xx можно завершить $T_2$ способами. Таким образом, $T_4=T_3+T_2=5$.
Вот эти пять покрытий: ||||, ||=, |=|, =||, ==.

Рассуждения для $n=4$ легко обобщить для любого значения $n \ge 2$: $T_n = T_{n-1} + T_{n-2}$.
Таким образом, получается такое же рекуррентное соотношение, как и для чисел Фибоначчи,
только начальные значения другие. Однако это --- последовательные члены ряда Фибоначчи
$F_1$ и $F_2$, так что последовательность $T$ совпадает с последовательностью Фибоначчи,
сдвинутой на одну позицию: $T_n = F_{n+1} \text{ при $n \ge 0$}$.

\subsection{Числа Фибоначчи}

Числа Фибоначчи определяются следующим рекуррентным соотношением:
\begin{align*}
F_0 &= 0, \\
F_1 &= 1, \\
F_n &= F_{n-1} + F_{n-2}, \text{ для $n>0$}
\end{align*}

Классическое приложение чисел Фибоначчи --- это определение размера популяции кроликов.
Если кролики становятся способными к размножению через месяц и каждая пара взрослых кроликов
производит на свет пару крольчат раз в месяц, то сколько пар кроликов будет в наличии
через $n$ месяцев.

\begin{task}
Написать программу, выводящую первые $n$ чисел Фибоначчи.
\end{task}

\begin{task}
Написать программу для разложения натурального числа на числа Фибоначчи.
\end{task}

\begin{task}
Начиная с какого числа Фибоначчи переполняются стандартные типы \texttt{int} и \texttt{long long}?.
\end{task}

Пример рекурсивного алгоритма для вычисления чисел Фибоначчи:

\begin{lstlisting}
int fib(int n)
{
    if (n == 0) return 0;
    if (n == 1) return 1;
    return fib(n-1) + fib(n-2);
}
\end{lstlisting}

\subsection{Покрытие прямоугольника $3\times n$}

Сколько существует способов покрыть костяшками домино $3 \times n$-прямоугольник
(обозначим это число $U_n$)? Пустое покрытие $U_0 = 1$. Для $n=1$ не существует допустимых
покрытий, поскольку одна $2\times 1$-плитка не заполняет $3 \times 1$-прямоугольник,
а для двух нет места.

Следующий случай, $n=2$, легко анализируется вручную: имеется три покрытия ${||}\atop{-}$,
${-}\atop{||}$ и $\equiv$, так что $U_2 = 3$. Для $n=3$ нет покрытий, так как его
площадь нечётна и поэтому ее не получится собрать из костяшек чётной площади.

Каждое непустое покрытие начинается с $\equiv$, L или Г. Но в последних двух случаях
не получится составить рекуррентность для $U_n$, так как правый край у них неровный.
Поэтому можно ввести дополнительную рекуррентность для покрытия. $V_n$  --- число
способов замостить $3\times n$-прямоугольник, оставив левый нижний угол пустым.
Аналогично $\Lambda_n$ --- число
способов замостить $3\times n$-прямоугольник, оставив левый верхний угол пустым.

Входящие в $V_n$ и $\Lambda_n$ покрытия могут начинаться двумя способами: с одной вертикальной
костяшки, которая выравнивает уже заполненный край и с трех костяшек, одна из которых сдвинута вправо.
Тогда,

\begin{align*}
U_0 &= 1 \\
U_1 &= 0 \\
U_n &= U_{n-2} + V_{n-1} + \Lambda_{n-1} \quad & \text{ для $n>1$}\\
V_0 &= 0 \\
V_1 &= 1 \\
V_n &= U_{n-1} + V_{n-2} & \text{ для $n>1$}\\
\Lambda_0 &= 0 \\
\Lambda_1 &= 1 \\
\Lambda_n &= U_{n-1} + \Lambda_{n-2} & \text{ для $n>1$}\\
\end{align*}

Очевидно, что $V_n = \Lambda_n$. Тогда рекуррентное соотношение можно упростить:

\begin{align*}
U_0 &= 1 \\
U_1 &= 0 \\
U_n &= U_{n-2} + 2V_{n-1} \quad & \text{ для $n>1$}\\
V_0 &= 0 \\
V_1 &= 1 \\
V_n &= U_{n-1} + V_{n-2} & \text{ для $n>1$}\\
\end{align*}

В результате должны получиться следующие значения:
$$
 \begin{array}{c|ccccccc}
      & 0 & 1 & 2 & 3 & 4  & 5  & 6  \\
  \hline
  U_n & 1 & 0 & 3 & 0 & 11 & 0  & 41 \\
  V_n & 0 & 1 & 0 & 4 & 0  & 15 & 0  \\
 \end{array}
$$

\begin{task}
Используя рекурсию, написать программу, для нахождения числа покрытий 
$3\times n$-прямоугольника костяшками домино.
\end{task}

Альтернативная рекуррентность: $D$ --- число покрытий ровного прямоугольника,
$\Gamma$ --- прямоугольника с одним выступающим с края квадратом.

\begin{align*}
D_0 &= 1 \\
D_1 &= 0 \\
D_n &= D_{n-2} + 2\Gamma_n \\
\Gamma_0 &= 0 \\
\Gamma_1 &= 0 \\
\Gamma_n &= D_{n-2} + \Gamma_{n-2} \\
\end{align*}

$$
 \begin{array}{c|ccccccc}
           & 0 & 1 & 2 & 3 & 4  & 5  & 6  \\
  \hline
  D_n      & 1 & 0 & 3 & 0 & 11 & 0  & 41 \\
  \Gamma_n & 0 & 0 & 1 & 0 & 4  & 0  & 15 \\
 \end{array}
$$

\subsection{Вычисление рекуррентностей с помощью рекурсии}

\begin{itemize}
\item Один ряд --- одна функция
\item Параметр функции --- индекс элементов последовательности
\item Каждая из формул заменяется веткой с оператором возврата значения
\item Формулы с постоянным индексом нужно помещать в начало функции,
      чтобы не было бесконечной рекурсии
\end{itemize}
